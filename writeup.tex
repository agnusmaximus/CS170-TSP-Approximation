% Created 2015-05-05 Tue 15:14
\documentclass[11pt]{article}
\usepackage[utf8]{inputenc}
\usepackage[T1]{fontenc}
\usepackage{fixltx2e}
\usepackage{graphicx}
\usepackage{longtable}
\usepackage{float}
\usepackage{wrapfig}
\usepackage{rotating}
\usepackage[normalem]{ulem}
\usepackage{amsmath}
\usepackage{textcomp}
\usepackage{marvosym}
\usepackage{wasysym}
\usepackage{amssymb}
\usepackage{hyperref}
\tolerance=1000
\usepackage[a4paper,margin=1cm,footskip=.5cm]{geometry}
\usepackage{listings}
\usepackage{amsmath}
\author{Name: Max Lam | Section: 109 \\ Partners: Maya Reddy, Sneha Sankavaram, Natasha Sandy \\ Team Name:LMMNRSSS | Login-Submitted-With:cs170-qg}
\date{\today}
\title{NPTSP Project}
\hypersetup{
  pdfkeywords={},
  pdfsubject={},
  pdfcreator={Emacs 24.3.1 (Org mode 8.2.10)}}
\begin{document}

\maketitle
\tableofcontents

\newpage


\section{Main Idea}
\label{sec-1}
Two stages
\begin{itemize}
\item Matlab integer linear programming to outright solve \textasciitilde{}85\% of the problems
\item Simulated annealing to approximate the ones that couldn't be solved
\item Verify 6-7 inputs by hand. This is self-explanatory -- some inputs have very clear patterns. SA was run to validate them.
\end{itemize}

\section{Matlab integer linear programming}
\label{sec-2}
Matlab provides a function called 'intlinprog' which uses the branch\&cut algorithm.
The branch\&cut algorithm works by using branch and bound, and linear programming
to compute upper and lower bounds to the solution, which prunes nodes of the search tree.
The Matlab guide for TSP was adopted for our project.
The link is here: \url{http://www.mathworks.com/help/optim/ug/travelling-salesman-problem.html}
The idea is that TSP can be formulated as a integer linear program such that every node
has the constraint that it has exactly 2 edges coming out of it. Then we want binary integer
indicators representing whether each edge should be included in the tour, for a minimum path.
We modified the 'tour' by using a dummy node which has a distance of 0 to every other node.
At the end, we remove the dummy node, thus creating a minimum path, instead of minimum tour.
The way the tutorial handles subtours is it repeatedly adds constraints whenever it detects
multiple subtours, then re-runs the integer linear program again and again. We adopt this method
to detect invalid paths that traverse more than 3 cities with the same color consecutively.
Thus, whenever we detect that the path violates the 'non-partisan' property, we re-run the integer
linear program with the added constraints. So the overall process is as follows:\\
\begin{enumerate}
\item Formulate NPTSP as integer linear program
\item Run linear integer program until only 1 subtour, adding constraints if not satisfied
\item Check if satisfies 'non-partisan'. If so, exit with optimal path, else go to 2.
\end{enumerate}

\section{Simmulated Annealing}
\label{sec-3}
Very bare-bones approach. Uses the widely known simulated annealing approach. We
have a temperature variable which indicates how likely we are to take a bad solution.
So, generate a random valid partisan path, keep randomly modifying it, and take a bad
solution sometimes to avoid local minima. Keep track of the best path so far, then at the
end output it. So the process is as follows:\\
\begin{enumerate}
\item Generate greedy path -- choose nearest neighbor with lowest cost. Do for all nodes
\item Loop with temperature
\item If temperature not the end temperature
\begin{enumerate}
\item Make random changes to the path
\item If new path is better than the old, replace it
\item If new path is worse than the old, replace with e$^{\text{-diffscore/temperature}}$ probability
\item Decrease temperature by factor c
\item Keep track of best path
\item Goto 3
\end{enumerate}
\item Output best path
\end{enumerate}

\newpage

\section{Input File Generation}
\label{sec-4}
We followed the 2-Opt algorithm and tried to generate cases
that would be hard to approximate with it. This means clumping together
nodes of the same color. Drew small cases by hand, then extended the cases
to be 50x50 by adding random weights for the remaining edges

\newpage

\section{Running the Code}
\label{sec-5}
The code is separated into several files, some were experimental and don't quite
fully work. The ones that don't quite work include the DP solution (which sometimes
gives a partisan path) and the python approximation one. Therefore, we will mainly
be focusing on the matlab code, and the C/C++ Simulated Annealing code, which was
the only code we used to solve the inputs.

\section{Matlab code}
\label{sec-6}
Run matlab, and within the matlab console cd into projbase/code/matlab.\\
  Within the matlab console run TSP(inputfile,1,60).
\begin{itemize}
\item First argument is the input file
\item Second argument is whether to debugoutput or not
\item Third argument is time limit in seconds
\end{itemize}

\section{Simulated annealing code}
\label{sec-7}
cd into projbase/code/SA.\\
  run make\\
  run ./a.out < inputfile

\newpage
\section{Resources}
\label{sec-8}
\begin{itemize}
\item Matlab \& libraries
\item Matlab TSP tutorial: \url{http://www.mathworks.com/help/optim/ug/travelling-salesman-problem.html}
\item C/C++ stdlibs
\end{itemize}
% Emacs 24.3.1 (Org mode 8.2.10)
\end{document}
